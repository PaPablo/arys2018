%%%%%%%%%%%%%%%%%%%%%%%%%%%%%%%%%%%%%%%%%%%%%%%%%%%%%%%%%%%%%%%%%%%%%
% LaTeX Template: Project Titlepage Modified (v 0.1) by rcx
%
% Original Source: http://www.howtotex.com
% Date: February 2014
% 
% This is a title page template which be used for articles & reports.
% 
% This is the modified version of the original Latex template from
% aforementioned website.
% 
%%%%%%%%%%%%%%%%%%%%%%%%%%%%%%%%%%%%%%%%%%%%%%%%%%%%%%%%%%%%%%%%%%%%%%

%-------------------------------------------------------------------------------
% TITLE PAGE
%-------------------------------------------------------------------------------

\documentclass[12pt]{report}
\usepackage[a4paper]{geometry}
\usepackage[myheadings]{fullpage}
\usepackage{fancyhdr}
\usepackage{lastpage}
\usepackage{graphicx, wrapfig, subcaption, setspace, booktabs}
\usepackage[T1]{fontenc}
\usepackage[font=small, labelfont=bf]{caption}
\usepackage{fourier}
\usepackage[protrusion=true, expansion=true]{microtype}
\usepackage[english]{babel}
\usepackage{biblatex}
\bibliography{references}
\usepackage{sectsty}
\usepackage{url, lipsum}
\usepackage{listings}
\usepackage{color}

\definecolor{mygray}{rgb}{0.9,0.9,0.9}

\lstset{
    backgroundcolor=\color{mygray}
}

\newcommand{\HRule}[1]{\rule{\linewidth}{#1}}
\onehalfspacing
\setcounter{tocdepth}{5}
\setcounter{secnumdepth}{5}

%-------------------------------------------------------------------------------
% HEADER & FOOTER
%-------------------------------------------------------------------------------
\pagestyle{fancy}
\fancyhf{}
\setlength\headheight{15pt}
\fancyhead[L]{Administración de Redes y Seguridad - 2018}
\fancyhead[R]{Toledo Margalef}
\fancyfoot[R]{Page \thepage\ of \pageref{LastPage}}


\begin{document}

\title{ \normalsize \textsc{Administración de Redes y Seguridad}
        \\ [2.0cm]
        \HRule{0.5pt} \\
        \LARGE \textbf{\uppercase{Trabajo Práctico 3}}
        \HRule{2pt} \\ [0.5cm]
        \normalsize \today \vspace*{5\baselineskip}}

\date{}

\author{
        Cátedra: \\
        Lic: Zappellini, Bruno Damian \\
\\
        Integrantes: \\
        Toledo Margalef, Pablo Adrian \\
        UNPSJB - Trelew}

\maketitle


\tableofcontents
\newpage

%-------------------------------------------------------------------------------
% Section title formatting
\sectionfont{\scshape}
%-------------------------------------------------------------------------------

%-------------------------------------------------------------------------------
% BODY
%-------------------------------------------------------------------------------

\section*{1. Footprinting}
\addcontentsline{toc}{section}{Footprinting}

\subsection*{¿Qué es footprinting?}

Footprinting \nocite{WikipediaFootprinting}, también conocido como \texttt{reconocimiento} es una técnica utilizada para recolectar información sobre dispositivos y su entorno. Dentro de los métodos que se pueden utilizar se encuentran:

\begin{itemize}
    \item Consultas DNS
    \item Identificación de sistema Operativo (nmap)
    \item Escaneo de puertos
    \item Querys de WHOIS
    \item google hacking
\end{itemize}

\subsubsection*{DIG y WHOIS}

Se utilizaran las siguientes dos organizaciones

\begin{itemize}
    \item Administración Federal de Ingresos Públicos
        \begin{itemize}
            \item \texttt{afip.gov.ar}
        \end{itemize}
    \item IBM
        \begin{itemize}
            \item \texttt{ibm.com} 
        \end{itemize}
\end{itemize}

\textbf{DIG}: es una herramienta para realizar peticiones a servidores de nombres. tomando un nombre de dominio y realizando el lookup correspondiente.

\textbf{WHOIS}: herramienta que busca un nombre de dominio en la base de datos de ls RFC 3912, la base de datos que almacena la correspondencia entre usuarios registrados y nombres de dominio.

Realizamos las consultas utilizando \texttt{afip.gov.ar} y listamos los resultados

\begin{itemize}
    \item name: ADMINISTRACION FEDERAL DE INGRESOS PUBLICOS
    \item IP (dns lookup): 200.1.116.6
    \item Fecha de registro:  1997-05-26 00:00:00
    \item Fecha expiración: 2019-06-25 00:00:00
    \item Servidor de nombres: ns1.afip.gov.ar (200.1.116.10/32)
    \item Registrar: nicar
\end{itemize}

Ahora con \texttt{ibm.com} 

\begin{itemize}
    \item name: IBM.COM
    \item IP (dns lookup): 129.42.38.10
    \item Fecha de registro: 1986-03-19T05:00:00Z
    \item Fecha expiración: 2019-03-20T04:00:00Z
    \item Servidor de nombres: EUR2.AKAM.NET
    \item Registrar: CSC Corporate Domains, Inc.
\end{itemize}

\subsection*{NETCRAFT - www.unp.edu.ar}

Realizamos la consulta en \texttt{netcraft.net} sobre \texttt{www.unp.edu.ar} y listamos algunos datos que se muestran allí.

\begin{tabular}{l | l}
    \hline 
    Site title & Universidad Nacional de la Patagonia San Juan Bosco \\ \hline
    Date first seen & June 1998 \\ \hline
    language & Spanish \\ \hline
    Description & Sitio web de la Universidad Nacional de la Patagonia San Juan Bosco. \\ \hline
    Netcraft Risk Rating & 7/10 \\ \hline
    Netblock Owner & Red de Interconexion Universitaria \\ \hline
    Nameserver & chenque.unp.edu.ar \\ \hline
    IP address & 170.210.88.21 (VirusTotal) \\ \hline
    DNS admin & hostmaster@unp.edu.ar \\ \hline
    Hosting company & unp.edu.ar \\ \hline
    Top Level Domain & Argentina (.edu.ar) \\ \hline
    Hosting country & AR \\ \hline
    
\end{tabular}

\subsection*{archive.org}

Este sitio ofrece, una snapshot del sitio que se busque. Proviendo las cualidades, casi, completas que nos ofrecía. De este modo, si en algún momento se dejó al descubierto alguna información de valor y se realizó la snapshot, ese dato está disponible, por más que se haya cambiado en el sitio real.

\subsection*{Fingerprinting}

\begin{itemize}
    \item \textbf{www.google.com.ar}: gws ()
    \item \textbf{www.ing.unp.edu.ar}: nginx/1.10.3
    \item \textbf{www.microsoft.com}: Apache
    \item \textbf{www.google.com.ar}: Microsoft-IIS/10.0
\end{itemize}

\section*{Sección 2}

\subsection*{Scanning}

Consiste en la búsqueda exhaustiva de diversas cuestiones a determinado nivel o niveles para encontrar vulnerabilidades.

El \textbf{escaneo de hosts} se realiza dentro una subred y permite enumerar los dispositivos que se encuentran conectados a ella. Teniendo como objetivo un equipo en particular, se puede realizar un \textbf{escaneo de puertos} de forma tal que se puede saber qué puertos se encuentran abiertos y disponibles para inciar una conexión.

Cuando se cuenta con red de Wi-Fi se puede realizar un \textbf{escaneo de redes Wi-Fi} para identificar las redes disponibles y poder hacer algún ataque a alguna en particular. Lo mismo se puede hacer cuando se tiene conectividad por bluetooh, listando los dispositivos disponibles para atacar.


\subsection*{Posibilidad de escaneo}

\begin{itemize}
    \item Sólo manipulando ARP: Escaneo de Hosts. Requiere estar en el mismo segmento de red.
    \item Sólo manipulando ICMP: Escaneo de puertos y host. No requiere estar en la misma red.
    \item Sólo manipulando TCP: Escaneo de puertos. No requiere estar en la misma red.
    \item Sólo manipulando UDP: Escaneo de puertos. No requiere estar en la misma red
    \item Interpretando tráfico: Escaneo de Wi-Fi, Bluetooth. Hosts y puertos. Se puede estar fuera de la red, en el caso de las radiofrecuencias. Salvo para LAN, ahí es mandatorio estar dentro de alguna red. 
\end{itemize}

\subsection*{Escaneo de puertos}

En la máquina virtual provsita por la cátedra, como primera medida, ponemos a funcionar el servicio de ssh, que atiende en el puerto 22 y corremos el comando netstat para verificar la existencia de puertos abiertos.

\begingroup
    \fontsize{8pt}{10pt}\selectfont
\begin{lstlisting}[breaklines=true]
root@kali:~# netstat -nltp4
Active Internet connections (only servers)
Proto Recv-Q Send-Q Local Address           Foreign Address         State       PID/Program name    
tcp        0      0 0.0.0.0:22              0.0.0.0:*               LISTEN      1645/sshd           
\end{lstlisting}
\endgroup

Luego, utilizando \texttt{nmap} escaneamos por puertos TCP que se encuentren abiertos.

\begingroup
    \fontsize{9pt}{10pt}\selectfont
\begin{lstlisting}[breaklines=true]
root@kali:~# nmap -sV localhost

Starting Nmap 7.40 ( https://nmap.org ) at 2018-10-15 18:22 EDT
Nmap scan report for localhost (127.0.0.1)
Host is up (0.0000020s latency).
Other addresses for localhost (not scanned): ::1
Not shown: 999 closed ports
PORT   STATE SERVICE VERSION
22/tcp open  ssh     OpenSSH 7.4p1 Debian 10 (protocol 2.0)
Service Info: OS: Linux; CPE: cpe:/o:linux:linux_kernel

Service detection performed. Please report any incorrect results at https://nmap.org/submit/ .
Nmap done: 1 IP address (1 host up) scanned in 0.36 seconds
\end{lstlisting}
\endgroup

Como se puede observar, el puerto 22 (propio de ssh) fue detectado por nmap.

\pagebreak

Seguidamente, le pedimos a \texttt{nmap}, explicitamente, que escanee los 65536 puertos disponibles.

\begingroup
    \fontsize{9pt}{10pt}\selectfont
\begin{lstlisting}[breaklines=true]
root@kali:~# nmap -p0-65535 localhost

Starting Nmap 7.40 ( https://nmap.org ) at 2018-10-15 18:52 EDT
Nmap scan report for localhost (127.0.0.1)
Host is up (0.0000020s latency).
Other addresses for localhost (not scanned): ::1
Not shown: 65535 closed ports
PORT   STATE SERVICE
22/tcp open  ssh

Nmap done: 1 IP address (1 host up) scanned in 0.57 seconds
\end{lstlisting}
\endgroup

\subsubsection*{Escaneo manual de puertos}

\textbf{hping3 -c 3 -S -p 80 localhost}: en la salida del analizador de protocolos se observa que el host responde con la bandera de \texttt{reset}. Indicando que el puerto está cerrado. En este caso se realiza un escaneo de tipo SYN. Ya que sólo se envía un paquete con la bandera \texttt{SYN}, imitando el inicio de una conexión.


\pagebreak

%-------------------------------------------------------------------------------
% REFERENCES
%-------------------------------------------------------------------------------

\newpage
\printbibliography

\end{document}

%-------------------------------------------------------------------------------
% SNIPPETS
%-------------------------------------------------------------------------------

%\begin{figure}[!ht]
%   \centering
%   \includegraphics[width=0.8\textwidth]{file_name}
%   \caption{}
%   \centering
%   \label{label:file_name}
%\end{figure}


%\begin{figure}[!ht]
%   \centering
%   \includegraphics[width=0.8\textwidth]{graph}
%   \caption{Blood pressure ranges and associated level of hypertension (American Heart Association, 2013).}
%   \centering
%   \label{label:graph}
%\end{figure}

%\begin{wrapfigure}{r}{0.30\textwidth}
%   \vspace{-40pt}
%   \begin{center}
%       \includegraphics[width=0.29\textwidth]{file_name}
%   \end{center}
%   \vspace{-20pt}
%   \caption{}
%   \label{label:file_name}
%\end{wrapfigure}

%\begin{wrapfigure}{r}{0.45\textwidth}
%   \begin{center}
%       \includegraphics[width=0.29\textwidth]{manometer}
%   \end{center}
%   \caption{Aneroid sphygmomanometer with stethoscope (Medicalexpo, 2012).}
%   \label{label:manometer}
%\end{wrapfigure}

%\begin{table}[!ht]\footnotesize
%   \centering
%   \begin{tabular}{cccccc}
%   \toprule
%   \multicolumn{2}{c} {Pearson's correlation test} & \multicolumn{4}{c} {Independent t-test} \\
%   \midrule    
%   \multicolumn{2}{c} {Gender} & \multicolumn{2}{c} {Activity level} & \multicolumn{2}{c} {Gender} \\
%   \midrule
%   Males & Females & 1st level & 6th level & Males & Females \\
%   \midrule
%   \multicolumn{2}{c} {BMI vs. SP} & \multicolumn{2}{c} {Systolic pressure} & \multicolumn{2}{c} {Systolic Pressure} \\
%   \multicolumn{2}{c} {BMI vs. DP} & \multicolumn{2}{c} {Diastolic pressure} & \multicolumn{2}{c} {Diastolic pressure} \\
%   \multicolumn{2}{c} {BMI vs. MAP} & \multicolumn{2}{c} {MAP} & \multicolumn{2}{c} {MAP} \\
%   \multicolumn{2}{c} {W:H ratio vs. SP} & \multicolumn{2}{c} {BMI} & \multicolumn{2}{c} {BMI} \\
%   \multicolumn{2}{c} {W:H ratio vs. DP} & \multicolumn{2}{c} {W:H ratio} & \multicolumn{2}{c} {W:H ratio} \\
%   \multicolumn{2}{c} {W:H ratio vs. MAP} & \multicolumn{2}{c} {\% Body fat} & \multicolumn{2}{c} {\% Body fat} \\
%   \multicolumn{2}{c} {} & \multicolumn{2}{c} {Height} & \multicolumn{2}{c} {Height} \\
%   \multicolumn{2}{c} {} & \multicolumn{2}{c} {Weight} & \multicolumn{2}{c} {Weight} \\
%   \multicolumn{2}{c} {} & \multicolumn{2}{c} {Heart rate} & \multicolumn{2}{c} {Heart rate} \\
%   \bottomrule
%   \end{tabular}
%   \caption{Parameters that were analysed and related statistical test performed for current study. BMI - body mass index; SP - systolic pressure; DP - diastolic pressure; MAP - mean arterial pressure; W:H ratio - waist to hip ratio.}
%   \label{label:tests}
%\end{table}
